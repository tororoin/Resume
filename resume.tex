%-------------------------
% Resume in Latex
% Author : Anthony Dsouza
% License : MIT
%------------------------

\documentclass[letterpaper,10.8pt]{article}

\usepackage{latexsym}
\usepackage[empty]{fullpage}
\usepackage{titlesec}
\usepackage{marvosym}
\usepackage[usenames,dvipsnames]{color}
\usepackage{verbatim}
\usepackage{enumitem}
\usepackage[pdftex]{hyperref}
\usepackage{fancyhdr}
\usepackage{bibentry}
\usepackage{color, soul}

\begin{filecontents}{publication.bib}
@INPROCEEDINGS{9807853,
	author={Dsouza, Anthony John and Rachel Kumar, Abigael and Wilson, Akhil Koshy and Deshmukh, Rupali},
	booktitle={2022 Second International Conference on Advances in Electrical, Computing, Communication and Sustainable Technologies (ICAECT)}, 
	title={SynthPipe : AI based Human in the Loop Video Dubbing Pipeline}, 
	year={2022},
	volume={},
	number={},
	pages={1-5},
	doi={10.1109/ICAECT54875.2022.9807853}}

\end{filecontents}

\pagestyle{fancy}
\fancyhf{} % clear all header and footer fields
\fancyfoot{}
\renewcommand{\headrulewidth}{0pt}
\renewcommand{\footrulewidth}{0pt}

% Adjust margins
\addtolength{\oddsidemargin}{-0.375in}
\addtolength{\evensidemargin}{-0.375in}
\addtolength{\textwidth}{1in}
\addtolength{\topmargin}{-.5in}
\addtolength{\textheight}{1in}

\urlstyle{rm}

\raggedbottom
\raggedright
\setlength{\tabcolsep}{0in}

% Sections formatting
\titleformat{\section}{
	\vspace{-3pt}\scshape\raggedright\large
}{}{0em}{}[\color{black}\titlerule \vspace{-5pt}]

%-------------------------
% Custom commands
\newcommand{\resumeItem}[2]{
	\item\small{
		\textbf{#1}{: #2 \vspace{-2pt}}
	}
}

\newcommand{\resumeItemWithoutTitle}[1]{
	\item\small{
		{\vspace{-2pt}}
	}
}

\newcommand{\resumeSubheading}[4]{
	\vspace{-1pt}\item
	\begin{tabular*}{0.97\textwidth}{l@{\extracolsep{\fill}}r}
		\textbf{#1} & #2 \\
		\textit{\small#3} & \textit{\small #4} \\
	\end{tabular*}\vspace{-5pt}
}


\newcommand{\resumeSubItem}[2]{\resumeItem{#1}{#2}\vspace{-4pt}}

\renewcommand{\labelitemii}{$\circ$}

\newcommand{\resumeSubHeadingListStart}{\begin{itemize}[leftmargin=*]}
	\newcommand{\resumeSubHeadingListEnd}{\end{itemize}}
\newcommand{\resumeItemListStart}{\begin{itemize}}
	\newcommand{\resumeItemListEnd}{\end{itemize}\vspace{-5pt}}

%-------------------------------------------
%%%%%%  CV STARTS HERE  %%%%%%%%%%%%%%%%%%%%%%%%%%%%


\begin{document}
	
	%----------HEADING-----------------
	\begin{tabular*}{\textwidth}{l@{\extracolsep{\fill}}r}
		\textbf{{\LARGE Anthony John Dsouza}} \\
		%\href{http://tororo.in}{Website: http://tororo.in} \\
		Linkedin : \href{https://www.linkedin.com/in/tororo/}{https://www.linkedin.com/in/tororo/} & Website : \href{https://tororo.in/}{www.tororo.in} \\
		Github : \href{https://github.com/tororoin}{https://github.com/tororoin} & Email : \href{mailto:aj.anthonydsouza@gmail.com}{aj.anthonydsouza[at]gmail.com}\\
		
	\end{tabular*}
	
	%-----------EDUCATION-----------------
	\section{Education}
	\resumeSubHeadingListStart
	%\resumeSubheading
	%  {Stony Brook University}{Stony Brook, NY}
	% {Masters in Computer Science;  GPA: 3.54}{Aug 2015 - Dec 2016}
	% 
	%  {\scriptsize \textit{Courses: Operating Systems, Analysis Of Algorithms, Artificial Intelligence, Machine Learning, Probability and Statistics and Network Security.}}
	
	
	%------------------------- EDUCATION 1 -------------------------
	\resumeSubheading
	{Universit{\"a}t des Saarlandes}{Saarbr{\"u}cken, Germany}
	{MSc. Language Science Technology (Computational Linguistics)}{October 2023 - Present} 	
	\vspace*{4pt}
	
	{\textit{\textbf{Courses:} \textcolor{blue}{\textbf{Computational Linguistics, Foundations of Linguistics, Statistics with R, Introduction to Python}}}}
	
	{\textit{\textbf{Seminars:} \textcolor{blue}{\textbf{Advances in Question Answering}}}}
	
	%------------------------- EDUCATION 2 -------------------------
	
	\resumeSubheading
	{Agnel Charities' Fr. C. Rodrigues Institute of Technology}{Navi Mumbai, India}
	{Bachelors of Engineering in Information Technology;  CGPA: 8.51/10.0 }{July 2018 - June 2022} 	\vspace*{4pt}

	{\textit{\textbf{Courses:} \textcolor{blue}{\textbf{Arificial Intelligence, Computer Organization and Architecure, Operating Systems, Automata Theory, Advanced Data Structures and Analysis of Algorithms, Information Retrieval Systems, Big Data, IoT.}}}}
	
	
	
	\resumeSubHeadingListEnd
	
	%
	%--------PROGRAMMING SKILLS------------
	\section{Skills Summary}
	\resumeSubHeadingListStart
	\resumeSubItem{Languages}{Python, C, SQL, Bash, \LaTeX}
	
	\resumeSubItem{Tools}{JAX/Flax/Haiku, PyTorch, TensorFlow/Keras, Accelerate, Transformers, TensorBoard, FastAPI, Docker, Triton, GIT, SQL}
	\resumeSubHeadingListEnd
	
	
	%-----------Goals----------------------
	
	\section{Goals}
	\resumeSubHeadingListStart
	%\resumeSubItem{Short Term Goals}{Study \textcolor{blue}{\textbf{multimodality and its applications in improving decision making process}} in autonomous systems, especially in healthcare, robotics, etc.}
	
	
	\resumeSubItem{Long Term Goals}{Research \textcolor{blue}{\textbf{Explainability in AI}} in decision making process.}
	\resumeSubHeadingListEnd
	
	%-----------EXPERIENCE-----------------
	%\section{Experience}
	%  \resumeSubHeadingListStart
	%   \resumeSubheading
	%  {VMware}{Palo Alto, CA}
	% {Member Of Technical Staff }{Feb 2017 -  Current}
	%\resumeItemListStart
	%        \resumeItem{Events and Alert Manager}
	%         {Network Fabric Controller is a logically centralized software controller to manage a distributed physical network fabric or a physical network underlay. Designed and developed a library which can be used by any services within Network Fabric Controller to generate events and raise alerts for NFC managed objects. The events and alerts are displayed on the NFC dashboard.}
	%        \resumeItem{Upgrade NFC}
	%       {Designed and developed an over-the-air and air-gapped upgrade mechanism that is used to upgrade the single node Network Fabric Controller cluster.}
	%      \resumeItem{Health Monitoring System}{Designed and developed a monitoring service which is responsible for monitoring the health of all the micro services running inside NFC cluster.}
	%     \resumeItem{CLI framework}{Developed an internal command line interface tool which provides a set of commands specific to Network Fabric Controller projects to get the system health, logs and current resource utilization. It can be easily extended to perform various other actions.}
	%    \resumeItem{Bootstrap NFC}{Network Fabric Controller is composed of several micro services deployed on the Kubernetes pods on a single-node cluster. Designed and implemented the bootstrapping mechanism to package all the services and deploy on the Kubernetes environment.}
	%   \resumeItem{Install/Upgrade/Uninstall NSX agent}{Worked on install, upgrade and uninstall mechanism of NSX agent on workload VMs deployed on NSX cross cloud environment.}
	%  \resumeItem{AppDiscovery}{Worked on application profiling feature which provides visualization and details of which processes inside a workload VM are communicating on the network.}
	%\resumeItemListEnd
	
	%    \resumeSubheading
	%		{Stony Brook University}{Stony Brook, NY}
	%		{Research Assistant - Prof. Erez Zadok }{May 2016 -  August 2016}
	%		\resumeItemListStart
	%       \resumeItem{System Call Trace Record/Replay}
	%          {Worked on building a trace replayer at system call level to reproduce system call operations that were captured during a specific workload using C, C++, DataSeries. Developed a wrapper class that makes C++ functions callable by strace C code.}
	%		\resumeItemListEnd
	
	%    \resumeSubheading
	%   {Samsung Research Institute}{Noida, India}
	%  {Software Developer Engineer}{Jun 2012 - July 2015}
	%    \resumeItemListStart
	%    \resumeItem{Android File System}{}
	%    \begin{description}[font=$\bullet$]
		%    \item {Involved in board bring-up activities for Android Smart phones based on Exynos and Broadcom chipsets on Android version 4.3 Jelly Bean to Android 5.0 Lollipop.}
		%   \item {Experienced in porting of File System (FAT, EXFAT, SDCARDFS, EXT4) on Samsung mobile’s proprietary platform.}
		%    \item {Enhanced performance of smart phones having low RAM by analyzing performance using blktrace and tuning kernel parameters. The code was merged in around 15 smart phones.}
		%  \end{description}
	%    \resumeItemListEnd
	%\resumeSubHeadingListEnd
	
	%\section{Current Projects}
	%\resumeSubHeadingListStart
	%\resumeSubItem{JAX based Gradient Boosting library}{JAX provides an ecosystem for easy distributed computing over GPUs and TPUs, with features like vmap, pmap, pjit and kernel fusion. A GBM written in JAX will help speed up training significantly. (July '22 to ongoing)}
	%\resumeSubItem{}{JAX/Flax/Haiku, PyTorch, TensorFlow/Keras, Transformers, TensorBoard, Docker, Triton, GIT, Matlab, SQL, MongoDB }
	%\resumeSubHeadingListEnd
	
	
	
	
	
	
	
	%-----------PROJECTS-----------------
	\section{Research Projects}
	
	\resumeSubHeadingListStart
	\resumeSubItem{\underline{Using CLIP to detect poachers }(\textcolor{blue}{Machine Learning, Multimodality, Information Retrieval, IoT, Anti-poaching)} }{Developed a solution to \textbf{detect poaching and illegal tree felling using multimodal models}. Given text prompts, the model could accurately detect poachers and tree fellers BEFORE the act. A node comprising of a low power SBC like Raspberry Pi Zero with camera, IR sensor, GPS and a PIR sensor is used to detect, capture images/videos and send data over to a backend server for further processing using trained CLIP model. \textbf{(Feb '22 to Mar '22)}}
	
	\vspace*{4pt}
	
	\resumeSubItem{\underline{Voice Deepfakes for Video Dubbing} (\textcolor{blue}{Machine Learning, Speech Processing, AI for Education, \textcolor{red}{Bachelor's Project})}}{ Developed a solution to \textbf{transcend linguistic barriers in education} by training ASR, NMT, Speech Synthesis model with desired voice and prosody so that learners are able to avail resources irrespective of the source language. \textbf{(Feb '21 to Mar '22)}}
	
	\vspace*{4pt}
	
	\resumeSubItem{\underline{AI based sub-millisecond poacher detection system} (\textcolor{blue}{Computer Vision, IoT, Anti-poaching, Object Detection)}}{Developed a solution to \textbf{detect poachers on the edge using quantized object detection model}. Deployed Raspberry Pi 3B+ nodes with PIR, IR and Camera modules detected poachers on device and communicated over a pub-sub network. \textbf{(Feb '21 to Mar '21)}}
	
	\vspace*{4pt}
	
	\resumeSubItem{\underline{DeepFake detection using CNN and Remote Photoplethysmography}  (\textcolor{blue}{Computer Vision, Fake News Detection)}}{Implemented a system to detect \textbf{image manipulation using remote photoplethysmography (rppg) and CNN}, based on the hypothesis that \textit{\textbf{GANs are superficial learners, and hence cannot accurately model data without artefacts.}} \textbf{(Jan '20 to Mar '20)}}
	
	%\vspace*{4pt}
	
	\resumeSubHeadingListEnd
	
	%-----------Awards-----------------

	\section{Publications}
	%\begin{description}[font=$\bullet$]
	%\item {SynthPipe : AI based Human in the Loop Video Dubbing Pipeline, 2022 Second International %Conference on Advances in Electrical, Computing, Communication and Sustainable Technologies %(ICAECT), 2022, pp. 1-5, doi: 10.1109/ICAECT54875.2022.9807853.} 
	%\item {Ranked first among batch of 60 students in my Computer Science Engineering Branch.}
	%\item {Ranked fifth among batch of 500 students at High School Level A.I.S.S.E 2005}
	%\end{description}
	\resumeSubHeadingListStart
	\resumeSubItem{SynthPipe : AI based Human in the Loop Video Dubbing Pipeline}{A. J. Dsouza, A. Rachel Kumar, A. K. Wilson and R. Deshmukh, 2022 Second International Conference on Advances in Electrical, Computing, Communication and Sustainable Technologies (ICAECT), 2022, pp. 1-5, doi: 10.1109/ICAECT54875.2022.9807853}
	\resumeSubHeadingListEnd
	%-------------------------------------------
	

	\section{Certifications}
		\resumeSubHeadingListStart
	\resumeSubItem{Structuring Machine Learning Projects}{\href{https://coursera.org/share/985e2fa446a8c2a301d90c507abc913f}{\textit{Coursera Certificate}}}
	
	\resumeSubItem{Natural Language Processing with Classification and Vector Spaces}{\href{https://coursera.org/share/2e073afe26926b2e0cedd9c4c60a4e59}{\textit{Coursera Certificate}}}
	
	\resumeSubItem{Natural Language Processing in TensorFlow}{\href{https://coursera.org/share/ebae442c18e776da47f3159495a5cc07}{\textit{Coursera Certificate}}}
	
	\resumeSubItem{Convolutional Neural Networks in TensorFlow}{\href{https://coursera.org/share/2bf11a455ddd16087f9245830ec29639}{\textit{Coursera Certificate}}}
	
	\resumeSubHeadingListEnd
\end{document}